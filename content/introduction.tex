% !TeX root = ../master_thesis.tex

\unnumberedchapter{Introduction}

The aim of the thesis is to research current state of industry, regulations and technology in order to examine requirements for a Chat-bot solution with Artificial Intelligence in a Commercial Banking Front Office and usefulness of such solution in a current market state.

The main hypothesis is: Commercial Banking clients are ready for Automated Front Offices in a form of Chat-bot.
To proof this thesis the descriptive method and survey will be used.

The main literature used are scientific papers that verify functionality of commercial banking. 
The fundamental papers of this work were "AI in banking: the reality behind the hype" written by Laura Noonan, "Artificial Intelligence and The Banking Industry’s \$1 Trillion Opportunity" written by Lisa Joyce and "AI Could Destroy Traditional Banking As We Know It" written by Jim Marous.
Those articles focus on occurring changes of banking by AI and its importance in near future. 

Extremelly important for this thesis were researches "Redefine Banking with Artificial Intelligence", "Chatbots are here to stay" and "Ready for Conversational Banking?" by Accenture.
Both of those define main concepts of Conversational Banking and major role of chatbots backed by AI in banking evolution 

Among the books the most noteworthy is "Bank 4.0. Banking Everywhere, Never at a Bank" written by Brett King.
This book is about future of banking.
Author farsees fundamental banking paradigm shift from products to delivery services in conjunction with AI and other smart technologies.

Chapter I is devoted to an overview of current banking system and influence of digital banking over it.
Additionally, mentioned chapter I contains analysis of various movements of banking towards openness, Open Banking and PSD2.

In chapter II is an overview of Artificial Intelligence in Commercial Banking including historical analysis and current application.

Chapter III concentrates over practical application of an AI in a Front-office in a form of a chat-bot and a noval Conversaional Banking paradigm.
In addition, chapter III proposes a strategy for a chat-bot solution development and integration in existing bank system.

Finally, results of econometric study are presented. 
Study shows respondents' satisfaction with existing front-office and their preparation towards chat-bot as an interlocutor.

The last section consists of thesis conclusions. 

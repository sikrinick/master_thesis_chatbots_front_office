% !TeX root = ../master_thesis.tex

\unnumberedchapter{Conclusions}

The main trend in commercial banking for the last decade is openness and availability.
Even though, from a regulatory perspective, forcing banks to allow third-party services to operate with a bank in a digital way is a form to divide natural monopolies, openness and availability is highly important on banking market by themselves.
The market of information systems, services and technologies is in all-time high and, obviously, affects such a giant as a banking market.

Nowadays, commercial banks do both front-office and back-office work.
Famous banking stability and guarantees are good for back-office, but can be a massive hurdle in a front-office.
Front-office, as a form of client interaction, has to change as fast as possible in order to achieve its client, client's needs and purposes.
Especially, when entire human ecosystem creates new forms of people communication and interaction.
Extrapolating last two decades we can expect future development of both banking openness and world digitalization.
Accordingly, banks have to adapt to those changes.

Currently, two trends, two forms of bank evolution are formulating — Bank-as-a-Service and Bank-as-a-Platform.
Bank-as-a-Service in a final form makes from a bank a back-office only construct, whose customers are various financial services, which can be risky, can experiment and are not too big to fall.
Although, this form reminds of factoring, it is not.
In spite, those third-party financial services can create their own front-offices, and can operate in fields, which are not available to common bank, in a such way that clients even don't know in which banks their accounts and loans are.
Bank-as-a-Platform, oppositely, is a center, which unites other third-party financial services under its hood and brand name.

The main subject of this study was to research current state of industry, regulations and technology in order to examine requirements for a chatbot solution with Artificial Intelligence in a Commercial Banking Front Office and usefulness of such solution in a current market state.

The main hypothesis was: Commercial Banking clients are ready for Automated Front Offices in a form a chatbot. 
To proof this research survey was conducted. 
The result proves that the hypothesis has to be rejected.
People are not ready and have negative subjective feelings towards chatbots.
Changing those feelings would require significant marketing costs.

However, the most valid injection of a chatbot solution would be in a hybrid approach.
Hybrid approach is a synergy between human employee and an automated chatbot system.
In hybrid approach robot recommends an answer to an operator based on existing knowledge base.
In this case, customer trusts and employee, but receives exact fast answer from an internal chatbot system.

Therefore, due to technology development, digitalizaton banks need new forms of interaction.
Moreover, younger generation is known to be more textual, messenger-friendly, and often voice interaction impacts negatively.
Among possibilities, one of the most efficient is a chatbot.
From customer perspective chatbot allows solving problems without spending time on waiting for a customer service and without repeating same question towards multiple consultants.
From the banking side, more simple chatbots decrease costs on customer service and open the possibility to transfer hired employees into less digital fields that require specialization.
Furthermore, the same instrument may significantly increase client engagement and offer growth mechanisms.
On the other hand, in order to achieve client engagement and use mentioned growth mechanisms, bank has to take a massive amount of risk and costs.

Regardless, banking sector should use openness trend and entire digital evolution in order to be in the market.
Moving towards Open Banking helps banking to share work and risks in client interaction.
Consequently, mentioned Machine Learning based chatbots can be done entirely by third-party providers and supported by them, if there are proper instruments done by bank that allows those to use.
On the other side, third-party providers may not be able to do it entirely, but can offer certain blocks of logic.
In this case, a bank has to build a chatbot and invest, but with the help of third-party partners.
For a bank mentioned system would be a beautiful piece of a technology, a state-of-art, even though it will require lots of investments and time.

In my opinion, division of labor in banking sector is inevitable.
Unfortunately, it is impossible to know which way would banks develop, towards Banks-as-a-Platform or towards Banks-as-a-Service.

In both options banks have to make actions towards smart textual forms of client communication.
The point is to develop solution iteratively.
Firstly, in all cases banks have to move to even Opener Banking and create even more extensive API.
After that, banks can choose which way do they want to go, either to use already existing solution and integrate with their systems, or to create something entirely new with the latest technologies available.
Integrating existing, more simple rule-based solutions without Artificial Intelligence is an efficient form of interaction due to the fact that entire chatbot industry is still emerging.
On the other side, creating own chatbot with Artificial Intelligence, Machine Learning and Natural Language Processing is an insecure choice, possible for large banks, but definitely extremely dangerous for small- and medium-sized banks.
Therefore, building chatbots via third-party development companies as solutions on premise would be the most balanced way of development of this interaction.

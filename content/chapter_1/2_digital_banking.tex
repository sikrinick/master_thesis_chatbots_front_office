% !TeX root = ../../master_thesis.tex

\section{Digital banking}

In general, Digital Banking is relatively new term and its definition is vaguely differed from virtual and online banking.
The concept of Digital Banking is based on digitalization of banking services — making those available online.
Currently, there are two major ways of definition — inclusive and exclusive.
Inclusive definition requires for banking services to be digitalized and entirely available online.
As a result, this definition requires for every single element to be digitalized.
Exclusive way requires banking services to be available exceptionally in digital channels, as a contrast to traditional banking, which is by default is not digitalized at all.
By uniting both definitions we can state, that digital banking is an ability to execute financial operations remotely with a usage of personal electronic devices due to development of bank's information technologies.
\cite{digital_banking_2020}

The most developed forms of digital banking are virtual banks, online banks and internet banking.
Virtual bank is a bank, which is accessible primarly via digital channels. 
Basically, it is a bank, which is entirely placed in internet.
In some cases it may have an office for customers, most likely for branding purposes or for complicated and conflict situations.

Nevertheless, there are the most radical banks, which exist exclusively in internet and provide services remotely.
It is possible to confuse those with online banks, which are usually just banks using internet as a form of client communication.
In its turn, internet banking is usually just an additional client interface as a form of a personal web app or mobile application for banking operations.

In the Digital Age of XXI Century, digitization of banking services is more of a business requirement in modern market and financial sector evolution, then some overhyped technology.
Changes, that will be brought by digitization offer benefits for both financial institutuions and customers.
Firstly, it brings efficiency, as it requires digitalization on a core level, which allows deriving from technology. 
Currently, banks assume being digital as a tool, and not a core banking feature.
Secondly, it brings cost efficiency. There are reports, which estimate that banks can increase EBITDA margins by up to 40\% on digital automation and removing middlewares.
\cite{rise_digital_bank_mckinsey}

Additionally, technology would allow to faster react on environmental changes on every level.
Banks, being slow responding to nature by nature, do not need to be that conservative on every level of operation. As for front office, bank has to be agile to changes on a customer market. At the same time, bank has to be agile enough to react to regulatory changes as fast as it can.
Furthemore, digital banking significantly changes levels and forms of competitiveness for customer base. For banks, it is much easier to increase customer engagement as it would be able to access various digital channels which customers are using. At the same time, it would be much harder to obtain new customers without providing active competition policy.

Obviously, this is extremelly positive from a customer perspective.
As a whole, customer benefits are more obvious.
Firstly, customers obtain more options, more choices, and are able to freely switch between them. This includes both banking products and banks themselves.
Secondly, digital products are much easier to use. Customer should not come to a branch office for basic banking operations. Moreover, for younger tech generations it is especially important and usually an important argument on choosing bank to partner with. 
As a last point, due to higher competition for a customer, the last may achieve various cost advantages for increasing engagement, as for example, cashbacks or personally offered low loan rates.
\cite{what_is_digital_banking}


Even though digital banking is considered relatively young, factically, it is a mature form of retail banking.
According to Deloitte, 81\% of the most developed digital banking peers are incumbents — banks with long-established position on the market.
Among digital services, those incumbents offer end-to-end support of opening and maintaining both transactional (debit card, credit card, currency and current accounts), saving \& investment (saving and term accounts, mutual funds) and credit (cash loan and overdraft) products digitally.
All main players offer API for developers and almost all of them offer FinTech accelerator program or hold hackathons. 
This significantly increases involvement of third-parties in digital banking, resulting in high level of interoperation, resulting in solutions in transactions and personal account management.
However, personally I assume that this target of digital development is based on a PSD2 requirements, as, for example, only 22\% of top banks offer possibility to add accounts from other banks, as interintegration between banks is not required by PSD2, even though it may positively affect bank level of customer engagement.
Moreover, banks with high level of digitization show positive difference in various KPIs comparing to incumbent peers with lower levels of digitization.
\cite{digital_banking_maturity}


As a next level of banking system development, digital banking adopters pioneer various technolgies and enablers. 
The most common and naive is white label banking. 
White labelling allows a bank, service provider, to provide banking business without product management and distribution. White label client, co-branding partner, may offer banking product by its name, while bank entirely takes responsibility for entire financial procedure.
The most known example of white labelling are branded credit cards.
As a next stage, banks can offer Banking-As-A-Service solutions. 
In this case, partners may use bank system as a logical and functional component, while entire product can be created by third-parties. It may be some sort of digital wallet, which requires bank to hold an account and to execute transactions, while an application for this wallet and entire registration and product distribution can be done by third-parties as middleware between bank and client.
The most demanding form of integration is Banking-as-a-Platform. Being a platform for thirdparties, bank allows using core banking systems as a base, which allows building not only products, but entire services on this base. 
The possibility to create such open banking systems are the main target of such initatives as UK Open Banking project and PSD2.
\cite{what_is_digital_banking}

In recent years, there has been an increasing need for more personalised and integrated services for banking customers, but traditional banks have not been able to meet these customers’ expectations.

Based on researches, 45\% of the American population, 62\% of the British population and 67\% of the Hong Kong population believe that banks are not meeting their needs.
\cite{wavestone_virtual_banking}

As a result, each region in the world (and their respective regulator) has been making efforts towards increasing innovation within their banking industry.

Many other regions in the world have been launching regulations or activities around virtual banking and open banking in order to promote the convergence of technology and banking in their own region.

However, each region has used a different approach with its own flavour, timing or implications.
This report aims at exploring whether virtual banking licenses 
or open banking regulation is the key to disrupt the retail banking industry, 
or whether it requires a bit of both. 

It also serves to explore and examine the different policies and regulations in place that disrupt the retail banking industry in the UK, Europe, US, Singapore and China whilst also looking at what traditional banks are doing to cope 
with these specified regulations as well as their respective success stories. 

As for Asian market, the most innovative and developing player is Hong Kong.
The Hong Kong Monetary Authority (HKMA), the key banking regulator, announced in September 2017 its intention to upgrade the current banking system to a new Smart Banking Era through the launch of several initiatives, including the Virtual Banking license (which would allow an entity to deliver retail banking services 
primarily through digital channels instead of physical branches)
and the Open API Framework (which is aimed at allowing thirdparty service providers (TSP) to connect to and contact data exchange to the banks’ IT systems).

These initiatives will accelerate the speed of innovation in the banking industry in Hong Kong. 
It will then evaluate the impact and effectiveness of the different systems 
and come to a consensus of the main common element that seems 
to drive the change in the retail banking ecosystem in each of these countries, 
and infer what it means and what would cause a similar disruption in the Hong Kong retail banking market.
\cite{wavestone_virtual_banking}

In Europe this is being developed by PSD2 directive and UK Open Banking Initiative and will be analyzed in next section, \hyperref[sec:psd2]{PSD2 and Open Banking}.

% !TeX root = ../master_thesis.tex

\unnumberedchapter{Introduction}

Nowadays, it is practically impossible to imagine life without online banking. Online banking is extremely comfortable for the clients because of its simplicity and speed of provided services. 
It is pretty obvious, that banks are striving to develop this field. The main priority is the comfort of a client, leading to development of Big Data as a main branch of improvements. 
Based on Gartner, around 34\% of banks were investing into Big Data technologies. 

Banks are storing everything — profiles, transactions and customer history, internal information. 
Repositories are literally inflated to petabytes of data. 
Artificial Intelligence in a connection with Machine Learning allows software to study client behavior and make decisions autonomously. 

However, in any case, at least for now, there should be a manager, who would follow and approve decisions that were made programmatically. 
Both Artificial Intelligence and Machine Learning have been being used for a long time, but it is reaching its peak with Big Data, with which it allows processing huge amounts of information quickly and efficiently. 
Consequently, it is critically important for the banking industry to rely on and collect more information about each client. 

The aim of the thesis is to research current state of industry, regulations and technology in order to examine requirements for a chatbot solution with Artificial Intelligence in a Commercial Banking Front Office and usefulness of such solution in a current market state.

The main hypothesis is: Commercial Banking clients are ready for Automated Front Offices in a form of a chatbot.
To proof this thesis the descriptive method and survey will be used.
The survey was conducted in order to determine current level of satisfaction with front-office. 
In entire survey 278 respondents were involved. 

The main literature used are scientific papers that verify functionality of commercial banking. 
The fundamental papers of this work were "AI in banking: the reality behind the hype" written by Laura Noonan, "Artificial Intelligence and The Banking Industry’s \$1 Trillion Opportunity" written by Lisa Joyce and "AI Could Destroy Traditional Banking As We Know It" written by Jim Marous.
Those articles focus on occurring changes of banking by AI and its importance in near future. 

Extremely important for this thesis were researches "Redefine Banking with Artificial Intelligence", "Chatbots are here to stay" and "Ready for Conversational Banking?" by Accenture.
Both of those define main concepts of Conversational Banking and major role of chatbots backed by AI in banking evolution 

Among the books the most noteworthy is "Bank 4.0. Banking Everywhere, Never at a Bank" written by Brett King.
This book is about future of banking.
Author foretells fundamental banking paradigm shift from products to delivery services in conjunction with AI and other smart technologies.

Chapter I is devoted to an overview of current banking system and influence of digital banking over it.
Additionally, mentioned chapter \Romannum{1} contains analysis of various movements of banking towards openness, Open Banking and PSD2.

In chapter \Romannum{2} is an overview of current application of Artificial Intelligence and Conversational Banking paradigm.

Chapter III concentrates over novel practical application of an AI in a Front-office in a form of a chatbot.
In addition, chapter \Romannum{3} proposes a strategy for a chatbot solution development and integration in existing bank system.

Finally, results of econometric study are presented. 
Study shows respondents' satisfaction with existing front-office and their preparation towards chatbot as an interlocutor.

The last section consists of thesis conclusions. 

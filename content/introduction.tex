% !TeX root = ../master_thesis.tex

\unnumberedchapter{Introduction}
Nowadays, it is practically impossible to imagine life without online banking.
Online banking is extremelly comfortable for the clients because of its simplicity and speed of provided services.
It is pretty obvious, that banks are striving to develop this field.
The main priority is the comfort of a client, leading to development of Big Data as a main branch of improvements.
Based on Gartner, around 34\% of banks were investing into Big Data technologies.

Banks are storing everything — profiles, transactions and customer history, internal information.
Repositories are literally inflated to petabytes of data.
Artificial Intelligence in a connection with Machine Learning allows software to study client behaviour and make decisions autonomously.

However, in any case, at least for now, there should be a manager, who would follow and approve decisions that were made programmatically.
Both Artificial Intelligence and Machine Learning have been being used for a long time, 
but it is reaching its peak with Big Data, with which it allows processing huge amounts of information quickly and efficiently.
Consequently, it is critically important for the banking industry to rely on and collect more information about each client. 

The reason for the development of the financial sector at the moment is its use of Big Data,
which allows you to provide a fully customized service for each client.
Information is 21st century gold and these technologies use it to provide services that a client already requires.
One of the priorities for the financial sector now is to collect information about each client to improve his experience.
The most commonplace example will be the operations at the ATM, which the client constantly performs.
The goal of the bank is to process information received about all operations and, on the next visit, 
immediately carry out the usual operation with one button, without searching and numbers.

That is why the International Data Corporation (IDC) forecasts Big Data Analytics profit growth to 
\$274 billion dollars by 2022 (against \$149 billion in 2017).
\cite{big_data_revenue_forecast}
The largest amount of investment goes to Big Data startups that are connected to the cloud and use a subscription system.
Also, many startups combine Big Data Analytics with Artificial Intelligence and Machine Learning 
to enhance the capabilities of this area. 

Information obtained using Big Data can be used to:
Creation and functioning of the engine, which determines the optimal place for opening physical banks.
The financial institution collects information about the most visited areas of the city, the time of visiting these areas,
stores in which their customers go, where the largest and least number of customers.
Using this data and analytics, you can determine the most profitable and
visited place to open a bank or any other financial institution. 

Risk management and fraud prevention, these are the 2 hottest topics for banks at the moment
and that’s why these projects were addressed first with innovative analytics technologies, Machine Learning and Big Data.
Banks calculate all possible options for risks and fraudsters and discard them at the first suspicion.
Extension of customer base interest in the bank's services: In addition to access to data on economic activity, the bank may also receive extraneous data, such as data from social networks or online behavior, to add this information to the ecosystem that surrounds the customer. 

By analyzing this information, which is located in Big Data, the bank discovers many new opportunities.
For example, if a user discusses the possibility of buying a new car in the comments,
then the bank can generate loan offers of the very machine that the client is dreaming of and send these offers instantly to him by e-mail.

Clients demand the most comfortable channels of communication that they use every day, such as social media, email, or instant messengers.
The bank should determine the priority channel of communications for certain group of clients 
and send all notifications, new offers and contact through them.
This will allow the client to feel that the bank is much closer to him and more service-friendly.
At the same time it will allow the bank to spend less on other channels of communication.
Moreover, it allows determining and predict when the client plans to change the financial institution by analyzing internal and external data about the client.
For example, if a client has not visited physical branches of a bank for a long time, does not visit a website and 
is subscribed to updates of other banks in social services, networks, then the chance that he leaves the bank can be predetermined. 

At such moment, it is important for the bank to promote to the client by recommending products 
or offers that the client is longing for at the moment.
One of the main problems of using Big Data, oddly enough, is the issue of ethical use of results, 
customers should feel a sense of security of their private life using the services of a bank.

The best option for using Big Data technology will be its use in the internal processes of the institution, 
without showing the most detailed information to the client.

This is achieved with a few steps, the first of which identifies the needs of certain customers,
and the next step masks these needs for mass mailing or possible communication scenarios, 
adjusting them to those that are already in the Big Data database. 

This thesis analyzes current state of banking in respect to various Open Banking initatives and PSD2.
Next, it analyzes Artificial Intelligence application in Commercial Banking and its history.
Considering the Conversational Banking trend and dialog-based communication need, author investigates existing forms of automatic conversational banking, voice and textual chat-bots.
Consequently, author proposes a front-office development strategy based on mentioned trends in a form of a chat-bot based on Artificial Intelligence, Machine Learning and Natural Language Programming.
Based on available statistics, author proposes ways of achieving strategy targets in accordance with existing market conditions.

The aim of the thesis is to research current state of industry, regulations and technology in order to examine requirements for a Chat-bot solution with Artificial Intelligence in a Commercial Banking Front Office and usefulness of such solution in a current market state.

The main hypothesis is: Commercial Banking clients are ready for Automated Front Offices in a form of Chat-bot with Artificial Intelligence and will positevely react to its injection into banking infrastracture.

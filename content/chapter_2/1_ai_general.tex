% !TeX root = ../../master_thesis.tex


\section{Artificial Intelligence in Operational bank activity}

Nowadays, banks are equipped with modern information and communication technologies.
At the same time mentioned technologies, including Fin-Tech software, have significant impact over existing financial institutions.
By surpassing operational limitations, those technologies become the main factor of transformation of both in a case of single bank, and in a case of entire banking sector. 
This results in a gradual increase of computerization of banking industry, leading to larger possibilities of application of artificial intelligence.

Significant volumes of information are being accumulated over financial markets resulting in data analysis being more and more relevant.
Some experts note, that markets are already emerging where data sharing is critical to competitive success and first movers are positioned to distinguish themselves by delivering better advice, constant presence, and curated ecosystems. 
Firms that lag behind are finding that their old strengths may not keep them as competitive as they once were.
\cite{ai_transform_disrupt}


\subsection{Artificial Intelligence in Investment Banking}

The pioneers of application of modern Artificial Intelligence and related technologies are, obviously, investment banking companies, which had to apply modern solutions in day trading algorithms. 
Those companies target Machine Learning and Natural Language Processing in order to use it for data, news and content analysis.
For them the most popular source of alternative data are news aggregators, expert networks and search query indexers.

Most of the asset managers and hedge funds specialists suppose that according to existing competitive dynamics, the trend of research disaggregation will continue even in regions not covered by MiFID II, legislative framework instituted by the European Union to regulate financial markets.

There is a popular opinion, that during researches investors would rely less on investment analysts. 
Some expect major changes in investment research market, as investors would need more data for support of AI and Machine Learning technologies.
On conducting researches, portfolio managers would rely less on investment analysts and more on internal solutions, data suppliers and solutions suppliers.
\cite{future_of_trading_technology_2024}


\subsection{Artificial Intelligence in Commercial Banking}

As for commercial banking, the status of AI integration highly depends on bank size.
All over the globe major financial players have been developing solutions based on Artificial Intelligence for the last 60 years with the various levels of success.
The situation has been especially progressive for the last 10 years.
In general, over 70\% of large global banks studied and have implemented AI for front-office or back-office functions.
\cite{deloitte_thriving_in_ai_era}

However, as for middle-size financial institutions, situation seems pessimistic.
While largest banks have been developing AI strategies, creating teams and projects in place by investing billions of dollars, for midsize banks AI was not even on the radar.

In comparison, while large banks have been investing majorly since 2016, in 2020 less than 20\% of midsize, or less, financial institutions invested in, implemented, or at least planned to apply AI.
What is even worse, only 2\% of those have deployed chat-bots, Machine Learning or other Artificial Intelligence technologies.
\cite{ai_transform_disrupt}

The main reason is in AI capital limitations.
Even though, those technologies can be independent of existing business structure, the last are highly dependent of capital investments.
Therefore, large financial institutions Banks that act now can capitalize on the power of automation and intelligence to truly transform their organizations.


Midsize and smaller banks and financial institutions have to compete with mega-sized counterparts without R\&D budgets.
What is even worse, the gap between large and non-large institutions only increase, because of how AI works.
The longer AI operates, the smarter and more useful it becomes.
As a result, the longer financial institutions wait, the harder it becomes to catch up. 
Financial institutions that start early gain a head start of months—even years—to gather data and “train” their self-learning, intelligent applications. 


Consequently, currently Artificial Technologies are available mostly to big players.

However, for midsize and smaller institutions not everything is lost, as there is a third-party option.
Machine Learning and Artificial Intelligence is highly used in start-ups, and application of solutions of those may save capital for research and development, and would allow applying tested solutions.
Secondly, most AI start-ups are small. 
Pilot programs and third-party innovation labs give banks and credit unions a chance to test, learn and refine your AI initiatives for a relatively small cost, before seeking funding for full-scale roll-outs.


However, one of the most important question of this analysis is what should banking acknowledge as Artificial Intelligence, what are its forms and possible use-cases in practice.

Currently, there is an extremely wide range of opinions.
Due to lack of unified theoretical platform of banking services, forecasts of development of banking institutions based on AI differ drastically and are based on business strategies of every single bank.
Nevertheless, there are 2 main routes of development.
First route is about focusing on cost reduction by workforce replacement and automatization of repeating routine operations without significant reformations of existing organizational structure.

On the other hand, there is a perspective in new digital technologies, which may allow inventing and introduce qualitatively different business models based on new market challenges, and, as a result, developing brand-new sources of income.
\cite{ai_reality_hype}

Nevertheless, in general, Artificial Intelligence in banks can be used in lots of areas.
From bank's perspective AI is needed from its ability to harness bank data in three key ways: to analyze, to act, and to improve by self-learning.

Systems of Artificial Intelligence were developed for automation of clerical workload, other routine paperwork, processing of various data sets.
In the digital age, banking transaction operations are just a form of abstraction over data transfer and storage.
Even though largest banks prefer to save more traditional organizational structure, they actively embed digital technologies in everyday practice.
Even though, banks prefer a less risky way, Artificial Intelligence application is applied in all bank fields, on all office levels and has to be precisely analyzed on each of those levels.



\subsection{AI in Back office}

Both Artificial Intelligence and Machine Learning may have significant influence on entire Back Office of Commercial banking.
Back office is known for enormous amount of repeating actions due to Execution, Clearing and Settlement process.
Therefore, the primary appliance of AI is in automatization of repeating routine operations.
Automated solutions, that automatically builds behavior patterns is ideal of Back Office, as it initiates high back-office efficiency via automation.

Ironically, this case is the case, when human intervention can impact more negatively, than positively.
For repeating actions, for example, calculations, in which there is no responsible decision-making, artificial intelligence suits the most.
“For repetitive tasks without variability (in middle office, in back end) for clearing/settlement/operational processes that are not particularly in need of smarts, 
then AI approaches are great,” says Pascal Bouvier, a venture partner at Santander InnoVentures, a fintech venture capital fund of the Spanish bank that invests in early stage Fin-Techs including those focused on AI.
\cite{ai_reality_hype}


Special attention of applied automatization is directed into processes, that require large amount of work, but offer low profitability.
McKinsey shows as an example of JP Morgan, which had started using chat-bots for IT service request automatization.
In 2017 1.7 million of requests where processed this way, which is equal to yearly full-time workforce of 40 employees.
\cite{ways_ai_transforming_bi}


In fact, it is possible for bank to transfer to robotic solutions such operations as:
\begin{itemize}[noitemsep]
    \item Payments processing of legal entities and individuals
    \item Processing of unidentified payments
    \item Customer data change based on statement
    \item Editing credit agreements based on individual statements
    \item Document processing
    \item Credit underwriting
\end{itemize}

Another popular application is processing of incoming documents.
Modern scanning programs can recognize standard documents and transfer them to performers.
Naturally, this kind of programs contain special filters, which direct documents, that do not fit into specific characteristics, for expert review.
Moreover, modern systems allow recognizing most typical types of documents and fill it using typical forms.
This allows a practical usage of these tools for legal office and even compliance control divisions.
As a result, activity of institutions become aligned to established regulations.
However, undeniable benefits of automatization based on artificial intelligence can be fully achieved and felt only in case of continuous update and upgrade of technological base.
\cite{banking_ai_revolution}

As an example of direct application, it is possible to refer to remittance matching solutions, that improve straight-through processing.
As an example, Deluxe offers AI based payment to remittance matching, which should shorten Days Sales Outstanding resulting in a same-day posting
\cite{deluxe_ai_remittance}

% if more needed: https://internationalbanker.com/finance/artificial-intelligence-storms-back-office/


\subsection{AI in Middle office}

In comparison to Back Office, where AI may be used primarly for infrastuctural changes and evolution, Middle office is open for the biggest cost cuts and optimizations.
According to researches, Middle Office alone can save \$217 billion, around 50\% of all cost savings available by AI for banks by 2023.
\cite{trillion_opportunity}

The reason for this is hidden in a definition of Middle office by itself.
Middle office is comparably new division of a commercial bank, and usually forms a bridge between customer flexible Front and strict and executive Back office.
Hence, Middle office operations by its nature are structured and documented, but have a certain level of statistical disturbance.
Thus, it makes Middle-office the most interesting and sensible target for AI development for commercial banking.

Importance is based on a combination of possible backfire of operation logic and unpredicatable patterns, that can form in everyday work.
This leads to, probably, the most obvious area for AI in Middle office — Credit Scoring.
AI in this case allows calculating credit scores not only based on known mathematical models, but to find out new, previously unknown patterns.
As an example, credit risks for private clients can be analyzed based on user's digital footprint, that can reach enormous amount of data.
Moreover, this may be true for clients, who have no credit history, or very old credit history, as it would allow to set a credit score based on a data of other clients.

Secondly, AI has a major usage in Anti-fraud.
Practically, it has the same reason to use as for Credit Scoring — it is possible to unite existing models with unpredictable patterns.
In this case model can analyze not just payment data, transaction history, transaction time and location, but in a various combinations based on possible patterns of transactions of other users.
Additionally, AI supports and provides face recognition systems.
As for ATMs, all actions can be guaranteed by face recognition.
It is true for Internet Banking for customers and CRM for bank employees as well.

Furthermore, the same reasoning is true for both KYC and AML compliance.
In general, Artificial Intellegence brings multiple possibilities for KYC and AML, such as unified pool of data, probabilistic matching, progressive evolution, self-training and pattern-based analytics in conjunction with rule-based.
Moreover, these features allow using AI in Risk Management, for credit and risk underwriting.

% proof: https://www.h2o.ai/resources/solution-brief/know-your-customer/

Based on existing estimations, around 20\% of digitalization projects are targeting cost reduction or productivity increase.
Most of those are targeting on potential risks recognition and minimization of risk consequences.
\cite{ai_reality_hype}

Artificial intelligence may frame out unlawful behavior, including non-stereotypical one.
By processing big data artificial intelligence may find out an evidence of fraud or illegally money laundering attempt.
The most simple way, but still significant, criminal pattern identification and revealing unlawful activity.
 
Artificial Intelligence can cover large databases and catch patterns, which hide from human attention.
Even more complicated approach of AI usage is educating to identifying market noises.
As a result, more advanced technologies allow discovering and determine people and companies with elevated risk for banks, allowing to build safer relationships with them in future.

Crimes in virtual banking are evaluated in \$600 billions yearly.
Artificial Intelligence and Machine Learning as main instruments for risk management in banking practice are mainly oriented on financial crime prevention.
This is much more efficient, comparing to complicated processes of damage compensation and financial crime evidence confirmation search.
Mastercard, for example, was able to reduce attacks on customer accounts by 80\%.
\cite{ways_ai_transforming_bi}
 


\subsection{AI in Front office}
\label{subsec:ai_front_office}

Numerous Financial Instituations are leveraging AI-driven chatbots and algorithms to support existing customer service channels with faster, more consistent intelligence.
One of the common thoughts on AI application in Front office are Biometrics.
Biometrics are widely represented by various recognition systems for much safer authentication.
Voice recognition is used to determine client by voice, facial recognition technologies, which are based on ML and AI, help to determine a client or to check correctness of client's photo data.
Similarly, Computer Vision technologies allows to determine the authenticity of physical signature.
However, it is more correct to consider biometrics as an input for Anti-fraud systems. 

As a minor use case, Front Office is interested in offering Personalized financial products, but this solution can be interesting for financial aggregators and Banks-as-a-Platform, but both those forms of financial institutions are comparably immature.

Even more rare is an application of AI in case of Robotic Advisors and Algorithmic trading on a consumer level.
However, it became more active since last year.

This is an alternative to communication with financial consultants in order to create and manage investment portfolios
with stocks, bonds and other assets.

In just a few minutes, according to the set parameters, the robot advisor can assemble a balanced, by industry and company, investment portfolio, taking into account the available investment amounts, with the optimal ratio of risk and profitability.

It is believed, that those systems can create accurate forecasts for stock market environment, due to possibilities of
automatic collection and analysis of the state of foreign exchange markets and latest economic news.
This allows client to invest into tools with the lowest risk.

Robo-advising became a powerful alternative to financial consultants in basic questions, related to banking, financial management and cash transactions.
The portfolio in the US financial markets, which is now being managed by robots, have already reached \$1 trillion in 2020
and rapidly increases, expecting to make up its volume up to \$2.85 trillion in 2025.
\cite{europarl_roboadvisors}

Going further, AI can be widely used not only for helping clients with investments, but in entire Individual Banking.
Standardized financial products and services for wide range of consumers cannot satisfy needs of modern client.
Modern clients require personified conditions for accounts, loans and other services. 
Without individual approach it is impossible to implement, therefore, one would need Artificial Intelligence as well.
Nowadays, every financial entity develops and offers between 10 and 20 financial products.
For developing such products banks need a team of professionals.
Developing hundreds of thousands of personified offers for every bank client is, by fact, impossible without help of Artificial Intelligence.
According to statistics, nearly every person has between 2 and 5 electronic devices, which can connect to internet, use various messengers or social networks.

Naturally, every internet user leaves enormous volume of data after himself.
Analyzing that personified data effectively allows creating various unique personified advices and offers.
Algorithms of Artificial Intelligence, theoretically, can collect client information, analyze and generate individual personified offer.

As a result, whole service industry becomes more and more personal.
Screening social networks profiles, collecting GPS or any other position data allows classifying client and to create his social portrait.
Obviously, from bank's side, it allows to effectively determine credit rating for every client.
At the same time, as an example, if one works in an agricultural field, bank can suggest and recommend various products connected with harvest insurance.
As for small business, analyzing supply chain and contragents of that business, bank can at the same time analyze seasonality of payments and more accurately predict time ranges, during which client liquidity can drastically decrease, or possible cash gaps.
This knowledge allows bank to operate more effectively and saves business from a useless hassle.

At the same time, case of Conversation Banking is extensively developing and has to be precisely analyzed.
Conversation Banking for the last 3 years became extremely popular.
Among the largest U.S. and international banks, the greatest focus today is on conversational interfaces, such as chatbots and virtual assistants.
\cite{deloitte_thriving_in_ai_era}

The main purpose of Conversation Banking is a client communication.
Nowadays, there are general purpose AI programs, that are able to speak to people.
Having a chat-bot call assistant for common problems allows bank to extend it into artificial personal assistant, including financial one, that can help a bank client using his and only his data.

Chatbots are one of the most effective ways to answer questions from employees and customers.
Client can call to a bank and talk about his problem, but due to the fact, that this problem can be categorized, a client can talk to bot, that can help client to solve his problem and tell about some problem related services.
Moreover, it can analyze client needs and immediately provide various financial recommendations.
Of course, that can be presented with any form of communication, both with SMS, internet messages or voice messages and calls.

But there is even more interesting point of view.
Based on the various researches, banks that are targeting teens and younger adults has to take to its concern, that younger people are not used to talk over phone, but to contact via messengers. 
Therefore, banks have to develop text chat-bots, and on various platforms, including bank's website, bank's mobile application, and chat-bots for application, where those people prefer to use, like Facebook.

In addition to chat-bots, banks can use voice assistants.
According to him, the assistant will help users in solving financial and everyday tasks — money transfer or reserving a table in a restaurant.

For example, Apple Siri, Amazon Alexa and Google Assistant are tightly coupled into everyday life of modern people and access to such fields of can result in an extremely high level of interaction between bank and client, and, therefore, increases client's loyalty. 

Bank's client can ask in this case various general questions and receive answers immediately.
For example, client can ask what payments were done with debit card last week, 
level of debt of a credit card and when fixed-term deposit ends.
As a result, client doesn't spend time looking for information on bank's webpage, sorting history in mobile or web app.
The assistant has another area of usage as well. 
For example, using an installed app, that is able to recognize products on store shelves, assistant can display cashback amount, which would be returned to the client after paying by credit card.

As a result, client communication is expected to outlive vitaly crucial challenges. 
Firstly, there is a growing competition over clients, both in attraction of new and in retention of old ones.
Secondly, ready-made technologies may allow to execute this much more effectively and with lower costs, then an operator.
Basically, almost all banks replace communication employees with a program — the chat-bot.
Usually this transformation is being done painlessly for a bank, as in most cases a foundation already exists.
In recent years operators were building client communication based on a script, scenario of a dialog.
In case of more complicated talk, operator have to transfer a talk to a manager with more responsibilities.
Based on research, modern chatting solutions can not be distinguished from an alive person by 60\% clients.
Chat-bots are attractive for business by providing consistent access to bank information with an immediate feedback, which is extremely important for the client.

Response delayed by 5 minutes decreases chances to create long-term bonds with a client by 10 times, resulting in building bonds with a bank, that answered the call, as in this case client feel himself important. 
Nevertheless, it is still important to leave a possibility to contact an alive employee, as 79\% of respondents would positively react on a conversation with a human,
while 74\% negatively react, if they would have a conversation with a machine, without a possibility to have a contact with a human.
However, modern programs may transfer communication to an employee, who has much more possibilities to handle a non-typical talk.
Client is important for a bank in case if he is using bank services on a regular basis, and constant contact is a reason of 70\% of purchases of banking services.
Chatbot, as a universal worker without breaks and competitors, especially for big banks with large client base, which bank has to be in a constant contact with.
Nowadays, such robots to a large extent are replacing call centers, mailing centers and in certain cases marketing divisions.

Swiss bank UBS, which is in Top 50 by assets, developed service Ask UBS to serve companies which specialize over managing client finance.
Offered service allows receiving advices in wide range of questions and financial markets analysis.
Moreover, it works as an educational service, clarifying definitions, giving definitions to acronyms and translating professional slang of the financial market.
According to UBS representative, bank intends to turn Ask UBS into a common instrument, which could be “secure, compliant, and trustable for clients.
\cite{ways_ai_transforming_bi}

Correctly developed program may not just execute multiple routine functions, but have an ability for self-education and self-correction.
Some specialists assume this would not be any burdensome development, if software developers didn't receive overcomplicated requirements.
However, extremely complicated solutions supporting multiple platforms and self-learning may be pretty expensive.
Therefore, most banks start exploring the digital world of chat solutions with simple virtual agents, which is mostly based on scripting and scenarios.

Another important direction of client communication improvement is the personalization of products and services.
Personalization is one of the main, if not the main, instruments for competition in banking sector.
Previous criteria of competitiveness, as price of banking products,  service delivery rate and accessibility, are fading out in significant extent, especially due to the fact of equalizing of competitors due to equivalent access to modern technologies, leaving no place for tangible superiority.

Competition over bank client moves towards digital solutions with the priority to develop client adapted, demand-based product, to form convincing recommendations, up to ready-made solutions, based on client needs, thoughts and expectations.
Banks obtain possibility to influence actively on client behavior by using large amounts of data.
By processing that data, artificial intelligence receive an option to offer client personified product on best suitable user conditions, as it, obviously, considers client's targets and financial possibilities.
It is common for such products to get ahead of client wishes, resulting in forming request by consulting and clarification.
\cite{ai_transform_disrupt}

Santander bank even instituted an award for developing a program, which would allow joining banking products with client needs.
\cite{ways_ai_transforming_bi}
In this part of banking the importance of consultants and intermediaries in relationships between banks and clients is increasing.

Naturally, conditions of digital economics allow for a robot to be the consultant, which would make recommendations about better products, institutions and services oriented on specific client.
This results in a trend of separating it into independent activities without binding to a single financial company.

Possible options of activity for general purpose solutions are much wider, as such solutions are not only based on bank products and services, but those can offer interaction with companies with more favorable conditions.
High level of personification of service and demand for intermediate service providers may even lead to the extintion of traditional forms of product offering, which is currently based on each product separately, for example, debit cards, payment systems, loans and saving accounts.
Instead of traditional offering, digitalization allows for development of universal financial service, which would implement all financial client requirements in a single integrated service, and what is more important, in way which would fit client needs the most.
One more step towards banking automatization is remote customer service for a certain range of operations.
Such possibility can be offered by a solution, identifying client by specific features — facial features, voice features or finger capillary network.

There are even more exotic options. 
For example, one Japanese researcher suggest using individual person's blinking behavior.
British bank Halifax conducts test of a special bracelet, which would register individual heartbeat features.
\cite{ways_ai_transforming_bi}

Some of those technologies have already been tested in production for financial operation. Most known example is the Apple Pay on iPhones, which uses facial recognition not only for phone unlocking, but also for executing operations via Apple Pay and other digital wallets.
Specialists claim that ways of person identification are becoming more complicated. 
On the other hand, those technologies at the same time become much more reliable and stop being exotic.

Google Intelligence predicts that by 2021 1.9 billion of bank clients would use biometric data for remote services.
However, there is still no large demand by banks for such remote forms of service, even though those technologies of identification have been developed long time ago and even tested by bank security services.
This is mainly due to lack of guarantee for operation to be voluntary, without any unlawful influence and there is no satisfying solution for such problem yet.
